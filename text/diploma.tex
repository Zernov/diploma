% Тут используется класс, установленный на сервере Papeeria. На случай, если
% текст понадобится редактировать где-то в другом месте, рядом лежит файл matmex-diploma-custom.cls
% который в момент своего создания был идентичен классу, установленному на сервере.
% Для того, чтобы им воспользоваться, замените matmex-diploma на matmex-diploma-custom
% Если вы работаете исключительно в Papeeria то мы настоятельно рекомендуем пользоваться
% классом matmex-diploma, поскольку он будет автоматически обновляться по мере внесения корректив
%

% По умолчанию используется шрифт 14 размера. Если нужен 12-й шрифт, уберите опцию [14pt]
%\documentclass[14pt]{matmex-diploma}
\documentclass[14pt]{matmex-diploma-custom}
\usepackage{graphicx}

\begin{document}
% Год, город, название университета и факультета предопределены,
% но можно и поменять.
% Если англоязычная титульная страница не нужна, то ее можно просто удалить.
\filltitle{ru}{
    chair              = {Кафедра информатики},
    title              = {Разработка системы автоматического анализа новостных публикаций на финансовом рынке},
    % Здесь указывается тип работы. Возможные значения:
    %   coursework - Курсовая работа
    %   diploma - Диплом специалиста
    %   master - Диплом магистра
    %   bachelor - Диплом бакалавра
    type               = {bachelor},
    position           = {студента},
    group              = 13.Б09-мм,
    author             = {Зернов Алексей Викторович},
    supervisorPosition = {к.\,ф.-м.\,н., доцент},
    supervisor         = {Григорьев Д.\,А.},
    reviewerPosition   = {позиция рецензента},
    reviewer           = {рецензент},
%   chairHeadPosition  = {д.\,ф.-м.\,н., профессор},
%   chairHead          = {Новиков Б.\,А.},
    university         = {Санкт-Петербургский Государственный Университет},
    faculty            = {Математико-механический факультет},
    city               = {Санкт-Петербург},
    year               = {2017}
}
\filltitle{en}{
    chair              = {Computer Science Department},
    title              = {Development of the automatic analysis system of a financial market's news publications},
    type               = {bachelor},
    position           = {student},
    group              = {13.Б09-мм},
    author             = {Alexey Zernov},
    supervisorPosition = {assistant professor},
    supervisor         = {Dmitry Grigoryev},
    reviewerPosition   = {reviewer position},
    reviewer           = {reviewer},
%   chairHeadPosition  = {professor},
%   chairHead          = {Boris Novikov},
    faculty            = {Faculty of Mathematics and Mechanics},
    city               = {Saint-Petersburg},
    year               = {2017}
}
\maketitle
\tableofcontents
% У введения нет номера главы
\section*{Введение}
not implemented


\bibliography{diploma.bib}
\end{document}
