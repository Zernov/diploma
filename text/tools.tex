\section{Инструменты и методологии}


\subsection{Natural Language Toolkit}

NTLK\footnote{\url{http://www.nltk.org}} является пакетом библиотек и программ для разработки программ на Python, работающих с естественным языком. Сопровождается обширной документацией, а также книгой\footnote{\url{http://www.nltk.org/book/}}, объясняющей основные концепции проблем, для решения которых предназначен данный пакет. NTLK~--- свободное программное обеспечение, то есть доступное бесплатно. Данный пакет подходит для таких областей как компьютерная лингвистика, эмпирическая лингвистика, когнитивистика, искусственный интеллект, информационный поиск и машинное обучение. NTLK используется преимущественно в качестве учебного пособия, индивидуального обучения или прототипирования и создания научно-исследовательских систем. Изначально пакет предназначен для англоязычных текстов, но имеется возможность обучения классификаторов для остальных языков.

\subsection{pymorthy2}

Pymorthy2\footnote{\url{https://pymorphy2.readthedocs.io/en/latest/index.html}} написан на языке Python и имеет следующие возможности:

\begin{itemize}

\item Приведение слова к нормальной форме

\item Ставить слово в нужную форму

\item Возвращать грамматическую информацию о слове

\end{itemize}



\subsection{Томита Парсер}

\subsection{Яндекс.Спеллер}

\subsection{OntosMiner}