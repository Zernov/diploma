% Тут используется класс, установленный на сервере Papeeria. На случай, если
% текст понадобится редактировать где-то в другом месте, рядом лежит файл matmex-diploma-custom.cls
% который в момент своего создания был идентичен классу, установленному на сервере.
% Для того, чтобы им воспользоваться, замените matmex-diploma на matmex-diploma-custom
% Если вы работаете исключительно в Papeeria то мы настоятельно рекомендуем пользоваться
% классом matmex-diploma, поскольку он будет автоматически обновляться по мере внесения корректив
%

% По умолчанию используется шрифт 14 размера. Если нужен 12-й шрифт, уберите опцию [14pt]
%\documentclass[14pt]{matmex-diploma}
\documentclass[14pt]{matmex-diploma-custom}
\usepackage{graphicx}

\begin{document}
% Год, город, название университета и факультета предопределены,
% но можно и поменять.
% Если англоязычная титульная страница не нужна, то ее можно просто удалить.
\filltitle{ru}{
    chair              = {Кафедра информатики},
    title              = {Разработка системы автоматического анализа новостных публикаций на финансовом рынке},
    % Здесь указывается тип работы. Возможные значения:
    %   coursework - Курсовая работа
    %   diploma - Диплом специалиста
    %   master - Диплом магистра
    %   bachelor - Диплом бакалавра
    type               = {bachelor},
    position           = {студента},
    group              = 13.Б09-мм,
    author             = {Зернов Алексей Викторович},
    supervisorPosition = {к.\,ф.-м.\,н., доцент},
    supervisor         = {Григорьев Д.\,А.},
    reviewerPosition   = {позиция рецензента},
    reviewer           = {рецензент},
%   chairHeadPosition  = {д.\,ф.-м.\,н., профессор},
%   chairHead          = {Новиков Б.\,А.},
    university         = {Санкт-Петербургский Государственный Университет},
    faculty            = {Математико-механический факультет},
    city               = {Санкт-Петербург},
    year               = {2017}
}
\filltitle{en}{
    chair              = {Computer Science Department},
    title              = {Development of the automatic analysis system of a financial market's news publications},
    type               = {bachelor},
    position           = {student},
    group              = {13.Б09-мм},
    author             = {Alexey Zernov},
    supervisorPosition = {assistant professor},
    supervisor         = {Dmitry Grigoryev},
    reviewerPosition   = {reviewer position},
    reviewer           = {reviewer},
%   chairHeadPosition  = {professor},
%   chairHead          = {Boris Novikov},
    faculty            = {Faculty of Mathematics and Mechanics},
    city               = {Saint-Petersburg},
    year               = {2017}
}
\maketitle
\tableofcontents
% У введения нет номера главы
\section*{Введение}
not implemented

\section{Обзор существующих решений}

\subsection{IBM: Watson Developer Cloud}

Данный сервис представляет из себя набор различных API\footnote{Application Programming Interface} для анализа текстовых, голосовых и визуальных данных. Среди их сравнительного большого количества я выделил несколько, о которых ниже будет рассказано подробнее.

\subsubsection{News Intelligence}

Одним из наиболее интересных сервисов для анализа новостных публикаций является приложение News Intelligence\footnote{\url{https://discovery-news-demo.mybluemix.net/}}.

Приложение предлагает ввести название интересующей Вас компании, после чего предоставляются следующие результаты:

\begin{itemize}

\item Наиболее упоминаемые сущности (люди, темы, компании)

\item Наиболее просматриваемые новостные публикации

\item Анализ тональности новостных публикаций из десяти случайных источников

\item Совместные упоминания и оценка их тональности

\end{itemize}

Важно отметить, что приложение News Intelligence является лишь примером использования данного инструмента, а не готовым продуктом.

\subsubsection{Social Customer Care}

Еще одним интересным примером использования Watson Developer Cloud является приложение Social Customer Care\footnote{\url{https://social-customer-care.mybluemix.net/}}. Оно осуществляет мониторинг социальных медиа, определяя потребности клиента или его запросы, а также автоматически отвечает в режиме реального времени.

\subsubsection{News Explorer}

Следующий пример~--- News Explorer\footnote{\url{http://news-explorer.mybluemix.net/}}. В данном приложении отображаются наиболее обсуждаемые запросы, наиболее часто встречающиеся совместные упоминания и список свежих новостей, разбитых по категориям.

В этом приложении также можно самостоятельно задать интересующий запрос, после чего будет отображена визуализированная карта взаимосвязей разных сущностей и новостных публикаций в виде графа.

\subsubsection{Investment Advisor}

И последним рассматриваемым примером из данной группы является приложение Investment Advisor\footnote{\url{http://investment-advisor.mybluemix.net/}}. В данном приложении есть две группы людей: инвесторы и представители компаний. На основе анализа личностных особенностей людей по их постам, строятся определенные рекомендации по вложениям и наиболее подходящим для сотрудничества представителям компаний.

\subsection{Microsoft: Text Analytics}

Text Analytics\footnote{\url{https://text-analytics-demo.azurewebsites.net/}} позволяет провести анализ тональности текста, выделив ключевые слова. Microsoft предоставляет набор методов API для работы с данным сервисом. Подробнее о работе с ними будет написано ниже.

\subsection{Медиалогия}

Медиалогия\footnote{\url{http://www.mlg.ru/}}~--- разработчик автоматической системы мониторинга и анализа СМИ в режиме реального времени.

Данная платформа предоставляет такие решения, как мониторинг СМИ компании (ее брендов, конкурентов и др.) и анализ СМИ и сообщений с использованием уникальной технологии лингвистического анализа текстов.

Из предоставленных примеров отчетов\footnote{В качестве образца был взял аналитический отчет компании <<Вымпелком>> в СМИ за \rom{2} квартал 2009 года} на сайте можно увидеть, что сервис учитывает следующее:

\begin{itemize}

\item \textbf{Количество упоминаний.} Отслеживается динамика по кварталам и месяцам. Отслеживая динамику по дням, платформа определяет наиболее заметные информационные поводы, вызывающие более сильный всплеск упоминаний. Также учитывается цитируемость, совместные упоминания и распределение по тематическим рубрикам.

\item \textbf{Качество упоминаний.} В рассмотренном отчете предоставлена информация о положительных и негативных сообщениях. Выделены пики и проанализирована связь со СМИ, которые способствуют больше благоприятному или отрицательному всплеску упоминаний.

\end{itemize}

Также в отчете были учтены распределения по уровням СМИ, по их географическому расположению, и прочее. Однако среди всего этого наиболее важным моментом является как раз анализ текста новости: является упоминание положительным или отрицательным, как это совместно упоминается с другими запросами и тому подобное.

\bibliography{diploma.bib}

\end{document}
