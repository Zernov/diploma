\section{Инструменты и методологии}

\subsection{Natural Language Toolkit}

NTLK\footnote{\url{http://www.nltk.org}} является пакетом библиотек и программ для разработки программ на Python, работающих с естественным языком. Сопровождается обширной документацией, а также книгой\footnote{\url{http://www.nltk.org/book/}}, объясняющей основные концепции проблем, для решения которых предназначен данный пакет. NTLK~--- свободное программное обеспечение, то есть доступное бесплатно. 

Данный пакет подходит для таких областей как компьютерная лингвистика, эмпирическая лингвистика, когнитивистика, искусственный интеллект, информационный поиск и машинное обучение. NTLK используется преимущественно в качестве учебного пособия, индивидуального обучения или прототипирования и создания систем, ориентированных на научно-исследовательскую деятельность.

Изначально пакет предназначен для англоязычных текстов, но имеется возможность обучения классификаторов для остальных языков.

\subsection{pymorphy2}

Pymorphy2\footnote{\url{https://pymorphy2.readthedocs.io/en/latest/index.html}}\cite{pymorphy2} написан на языке Python и имеет следующие возможности:

\begin{itemize}

\item Приведение слова к нормальной форме

\item Ставить слово в нужную форму

\item Возвращать грамматическую информацию о слове

\end{itemize}

Распространяется pymorphy2 под лицензией MIT\footnote{\url{https://opensource.org/licenses/MIT}}, если используется в научной работе.

\subsection{Томита-парсер}

Томита-парсер\footnote{\url{https://tech.yandex.ru/tomita/}} способен извлекать структурированные данные из текстов на естественном языке. Как и почти во всех инструментах, рассматриваемых в данном разделе, Томита-парсер ориентирован преимущественно на русскоязычные тексты. В нем используются контекстно-свободные грамматики и словари ключевых слов. Код проекта\footnote{\url{https://github.com/yandex/tomita-parser/}} находится в свободном доступе.

\subsection{Яндекс.Спеллер}

Яндекс.Спеллер\footnote{\url{https://tech.yandex.ru/speller/}} выполняет задачу проверки орфографии в текстах на английском, русском и украинском языках. Для этого используется орфографический словарь. К тому же, предоставлен набор API методов для реализации данной проверки разработчиками сайтов или приложений.

\subsection{OntosMiner}

OntosMiner\footnote{\url{http://my-eventos.com/solution/ontosminer/}} является решением компании Eventos\footnote{\url{http://my-eventos.com/solution/ontosminer/}}, занимающейся в большей степени разработкой продуктов в области лингвистического анализа текстовой информации, кластеризацией и классификацией информации. Конкретно OntosMiner является целой комплексной системой, дающей возможность распозавания связей между сущностями в текстах на естественной языке. Также, она позволяет определеять общую тональность текста.