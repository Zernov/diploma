\section{Финансовый рынок}

В данном разделе будет представлен краткий обзор основных терминов, связанных с самим финансовым рынком, его структурой и основными участниками. Более подробная информация может быть получена в \cite{financial_market}.

\subsection{Определение}

В более общем виде \textbf{финансовый рынок}~--- совокупность экономических связей его участников, касающихся создания, поддержания и обращения капитала. Финансовый рынок является довольно абстрактным термином, и под ним часто подразумеваются более конкретные: рынок купонных и бескупонных облигаций, рынок акций (или фондовый рынок) или валютный рынок. Не смотря на выделение составляющих, каждая из них является частью единого механизма, в котором финансы перемещаются между каждым из конкретных рынков.

Каждый из финансовых рынков является рынком посредников между начальными владельцами финансов и их конечными пользователями. Если рынок основывается на финансах как на капитале, он называется фондовым рынком, и именно в этой роли выступает как составная часть всего финансового рынка.

В России финансовые рынки имеют следующие критерии, влияющие на их деятельность:

\begin{itemize}
\item Инвестиции в экономику страны
\item Международные рынки, влиние тенденций глобализации
\item Современные компьютерные технологии, уровень комьютерной и информационной развитости участников рынков
\end{itemize}

\subsection{Структура}

Финансовый рынок может быть:

\begin{itemize}
\item Первичным или вторичным
\item Организованным или неорганизованным
\item Биржевым или внебиржевым
\item Традиционным или компьютеризированным
\item Кассовым или срочным
\end{itemize}

\textbf{Первичный рынок} обеспечивает выход ценных бумаг в оборот, это своеобразное <<производство>> ценных бумаг. На \textbf{вторичном рынке} в обороте находятся уже выпущенные ранее ценные бумаги. Вторичный рынок представляет из себя совокупность всех операций с данными ценными бумагами, в результате которых они переходят от одних владельцев к другим.

\textbf{Организованный рынок} отличается от \textbf{неорганизованного рынка} тем, что в первом имеются единые для всех участников рынка правила, за соблюдением которых следят организаторы. Во неорганизованном рынке соблюдение единых правил для всех участников рынка не гарантируется.

\textbf{Биржевой рынок}~--- такой рынок, на котором в качестве инструмента торговли используется аукцион. Руководителем же является некоторый специалист, например, NYSE\footnote{New York Stock Exchange~--- Нью-Йоркская фондовая биржа} или AMEX\footnote{(American Stock Exchange - Американская фондовая биржа}. На \textbf{внебиржевых рынках} торги организуются при помощи электронных систем. Руководят торгами маркет-мейкеры.

\textbf{Срочный рынок} чаще всего подразумевает отложенное исполнение сделки, в отличие от \textbf{кассового рынка}, когда сделки исполняются сразу. Обычно традиционные ценные бумаги (акции, облигации) идут в оборот на кассовом рынках, а контракты на производные инструменты рынка ценных бумаг~--- на срочных.

\subsection{Участники}

\textbf{Участники} рынка ценных бумаг~--- это физические лица или компании, которые продают или приобретают ценные бумаги, обеспечивают их оборот или расчеты по ним. 

Основными участниками рынка выступают \textbf{эмитенты}, выпускающие акции или облигации, с помощью которых привлекают финансирование, а также размещающие свободные на данный момент денежные средства. Эмитентами могут быть государство, субъекты государства или коммерческие предприятия. Целью эмитентов на первичном рынке является размещение запланированного транша по максимальной цене.

\textbf{Инвестор}~--- лицо, заинтересованное во вложении капитала в ценные бумаги. Их целью является как можно более выгодная покупка ценных бумаг максимально перспективных компаний.